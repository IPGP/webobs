%%%%%%%%%%%%%%%%%%%%%%%%%%%%%%%%%%%%%%%%%%%%%%%%%%%%%%%%%%%%%%%%%%%%
%%%%%%%%%%%%%%%%%%%%%%%%%%%%%%%%%%%%%%%%%%%%%%%%%%%%%%%%%%%%%%%%%%%%

\chapter{Appendix}

% ==================================================================

\begin{table}[htp]
\caption{Data raw formats for PROCS}
\label{rawformats}
\lstinputlisting{../../CODE/etc/rawformats.conf}
\end{table}


% ==================================================================
\begin{table}[htp]
\caption{Time scale keys (\textit{x} is an integer).}
\label{timescales}
\begin{center}
\begin{tabular}{|c|l|}
\hline
\textbf{Key} & \textbf{Time scale}\\
\hline
x\texttt{s}  & x seconds\\
x\texttt{n}  & x minutes\\
x\texttt{h}  & x hours\\
x\texttt{d}  & x days\\
x\texttt{w}  & x weeks\\
x\texttt{m}  & x months\\
x\texttt{y}  & x years\\
x\texttt{l}  & last x data\\
\texttt{r}x  & reference to date \wokey{REFx\_DATE}\\
\texttt{all}  & all data\\
\hline
\end{tabular}
\end{center}
\end{table}

% ==================================================================
\begin{table}[htp]
\caption{Marker type list}
\label{markertype}
\begin{center}
\begin{tabular}{|c|l|}
\hline
\textbf{Symbol} & \textbf{Marker}\\
\hline
\texttt{.}  & Point\\
\texttt{+}  & Plus sign\\
\texttt{*}  & Asterisk\\
\texttt{x}  & Cross\\
\texttt{o}  & Circle\\
\texttt{s}  & Square\\
\texttt{d}  & Diamond (vertical rhombus)\\
\texttt{\textasciicircum} & Upward-pointing triangle\\
\texttt{v}  & Downward-pointing triangle\\
\texttt{\textgreater} & Right-pointing triangle\\
\texttt{\textless} & Left-pointing triangle\\
\texttt{p}  & Five-pointed star (pentagram)\\
\texttt{h}  & Six-pointed star (hexagram)\\
\hline
\end{tabular}
\end{center}
\end{table}


% ==================================================================
\begin{table}[htp]
\caption{Line Style list}
\label{linestyle}
\begin{center}
\begin{tabular}{|c|l|}
\hline
\textbf{Symbol} & \textbf{Line style}\\
\hline
\texttt{-}  & Solid line\\
\texttt{--}  & Dashed line\\
\texttt{:}  & Dotted line\\
\texttt{-.}  & Dash-dot line\\
\hline
\end{tabular}
\end{center}
\end{table}


% ==================================================================
\begin{table}[htp]
\caption{Some basic R,G,B colors}
\label{rgbcolors}
\begin{center}
\begin{tabular}{|c|l|}
\hline
\textbf{R,G,B} & \textbf{Color}\\
\hline
\texttt{0,0,0}  & Black\\
\texttt{1,1,1}  & White\\
\texttt{1,0,0}  & Red\\
\texttt{0,1,0}  & Green\\
\texttt{0,0,1}  & Blue\\
\texttt{1,1,0}  & Yellow\\
\texttt{1,0,1}  & Magenta\\
\texttt{0,1,1}  & Cyan\\
\hline
\end{tabular}
\end{center}
\end{table}


% ==================================================================
\begin{table}[htp]
\caption{Some built-in colormaps}
\label{colormaps}
\begin{center}
\setlength{\fboxsep}{1pt}
\begin{tabular}{|c|l|}
\hline
\textbf{Name} & \textbf{Description}\\
\hline
\texttt{jet}  & \fbox{\includegraphics[width=3cm]{figures/jet.png}}\\
\texttt{hsv}  & \fbox{\includegraphics[width=3cm]{figures/hsv.png}}\\
\texttt{landcolor}  & \fbox{\includegraphics[width=3cm]{figures/landcolor.png}}\\
\texttt{seacolor}  & \fbox{\includegraphics[width=3cm]{figures/seacolor.png}}\\
\texttt{white}  & \fbox{\includegraphics[width=3cm]{figures/white.png}}\\
\texttt{gray}  & \fbox{\includegraphics[width=3cm]{figures/gray.png}}\\
\texttt{hot}  & \fbox{\includegraphics[width=3cm]{figures/hot.png}}\\
\texttt{cool}  & \fbox{\includegraphics[width=3cm]{figures/cool.png}}\\
\texttt{bone}  & \fbox{\includegraphics[width=3cm]{figures/bone.png}}\\
\texttt{pink}  & \fbox{\includegraphics[width=3cm]{figures/pink.png}}\\
\texttt{copper}  & \fbox{\includegraphics[width=3cm]{figures/copper.png}}\\
\texttt{spring}  & \fbox{\includegraphics[width=3cm]{figures/spring.png}}\\
\texttt{summer}  & \fbox{\includegraphics[width=3cm]{figures/summer.png}}\\
\texttt{autumn}  & \fbox{\includegraphics[width=3cm]{figures/autumn.png}}\\
\texttt{winter}  & \fbox{\includegraphics[width=3cm]{figures/winter.png}}\\
\hline
\end{tabular}
\end{center}
\end{table}


% ==================================================================
\begin{table}[htp]
\caption{Date string format list. ``Datenum'' column indicates if the format is valid for date/time string as key value}
\label{datestr}
\begin{center}
\begin{tabular}{|c|c|l|l|l|}
\hline
\textbf{Number} & \textbf{Datenum} & \textbf{String} & \textbf{Example} & \textbf{Comment}\\
\hline
\texttt{-1}		&		& automatic					&	& default value\\
\texttt{0}		& OK	& 'dd-mmm-yyyy HH:MM:SS'	& 01-Mar-2000 15:45:17	&\\
\texttt{1}		& OK	& 'dd-mmm-yyyy'				& 01-Mar-2000	&\\
\texttt{2}		&		& 'mm/dd/yy'				& 03/01/00	&\\
\texttt{3}		&		& 'mmm'						& Mar	&\\
\texttt{4}		&		& 'm'						& M	&\\
\texttt{5}		&		& 'mm'						& 03	&\\
\texttt{6}		&		& 'mm/dd'					& 03/01	&\\
\texttt{7}		&		& 'dd'						& 01	&\\
\texttt{8}		&		& 'ddd'						& Wed	&\\
\texttt{9}		&		& 'd'						& W	&\\
\texttt{10}		&		& 'yyyy'					& 2000	&\\
\texttt{11}		&		& 'yy'						& 00	&\\
\texttt{12}		&		& 'mmmyy'					& Mar00	&\\
\texttt{13}		&		& 'HH:MM:SS'				& 15:45:17	&\\
\texttt{14}		&		& 'HH:MM:SS PM'				& 3:45:17 PM	&\\
\texttt{15}		&		& 'HH:MM'					& 15:45	&\\
\texttt{16}		&		& 'HH:MM PM'				& 3:45 PM	&\\
\texttt{17}		&		& 'QQ-YY'					& Q1-96	&\\
\texttt{18}		&		& 'QQ'						& Q1	&\\
\texttt{19}		&		& 'dd/mm'					& 01/03	&\\
\texttt{20}		&		& 'dd/mm/yy'				& 01/03/00	&\\
\texttt{21}		&		& 'mmm.dd,yyyy HH:MM:SS'	& Mar.01,2000 15:45:17	&\\
\texttt{22}		&		& 'mmm.dd,yyyy'				& Mar.01,2000	&\\
\texttt{23}		& OK	& 'mm/dd/yyyy'				& 03/01/2000	&\\
\texttt{24}		&		& 'dd/mm/yyyy'				& 01/03/2000	&\\
\texttt{25}		&		& 'yy/mm/dd'				& 00/03/01	&\\
\texttt{26}		&		& 'yyyy/mm/dd'				& 2000/03/01	&\\
\texttt{27}		&		& 'QQ-YYYY'					& Q1-1996	&\\
\texttt{28}		&		& 'mmmyyyy'					& Mar2000	&\\
\texttt{29}		& OK	& 'yyyy-mm-dd'				& 2000-03-01	& ISO 8601\\
\texttt{30}		& OK	& 'yyyymmddTHHMMSS'			& 20000301T154517	& ISO 8601\\
\texttt{31}		& OK	& 'yyyy-mm-dd HH:MM:SS'		& 2000-03-01 15:45:17	&\\
\hline
\end{tabular}
\end{center}
\end{table}


% ==================================================================
\begin{table}[htp]
\caption{TeX stream modifiers and main escape characters allowed in titles and labels when specified, in addition to standard Greek letters and mathematical symbols.}
\label{texcommands}
\begin{center}
\begin{tabular}{|c|l|}
\hline
\textbf{TeX Modifier} & \textbf{Comment}\\
\hline
\texttt{\^}  & Superscript or exponent (use braces for more than 1 character)\\
\texttt{\_}  & Subscript (use braces for more than 1 character)\\
\texttt{$\backslash$bf}  & Bold font\\
\texttt{$\backslash$it}  & Italic font\\
\texttt{$\backslash$rm}  & Normal font\\
\texttt{$\backslash$fontname\{fontname\}}  & Specify the name of the font family to use\\
\texttt{$\backslash$fontsize\{fontsize\}}  & Specify the font size in FontUnits\\
\texttt{$\backslash$color(colorSpec)}  & Specify color for succeeding characters\\
\hline
\texttt{$\backslash$backslash}  & backslash character\\
\texttt{$\backslash$\{} or \texttt{$\backslash$lbrace}  & left brace character\\
\texttt{$\backslash$\}} or \texttt{$\backslash$rbrace}  & right brace character\\
\texttt{$\backslash$\_}  & underscore (low line) character\\
\texttt{$\backslash$\^} or \texttt{$\backslash$hat}  & hat character\\
\hline
\end{tabular}
\end{center}
\end{table}

% ==================================================================

\begin{table}
\caption{Suggestion for NODE ID code: \texttt{\textbf{N D T S S S S}}}
\label{nodeidcodes}
\begin{center}
\begin{tabular}{|c|c|l|}
\hline
\textbf{Letter} & \textbf{Code} & \textbf{Comment}\\
\hline
\texttt{N} = Network         & \texttt{I}	&	IPGP\\
      			             & \texttt{G}	&	OVSG\\
      			             & \texttt{M}	&	OVMP\\
      			             & \texttt{R}	&	OVPF\\
      			             & \texttt{P}	&	PVMBG\\
\hline
\texttt{D} = Domain          & \texttt{S}	&	Seismology\\
      			             & \texttt{D}	&	Deformations\\
      			             & \texttt{G}	&	Geophysics\\
      			             & \texttt{C}	&	Chemistry\\
      			             & \texttt{I}	&	Imagery\\
      			             & \texttt{M}	&	Meteorology\\
      			             & \texttt{P}	&	Phenomenology\\
      			             & \texttt{A}	&	Acquisition\\
\hline
\texttt{DT} = Technique      & \texttt{SB}	&	Broad-band\\
      			             & \texttt{SZ}	&	Short-period\\
      			             & \texttt{DC}	&	Continuous GPS\\
      			             & \texttt{DT}	&	Tilmetry\\
      			             & \texttt{DD}	&	Distancemetry\\
      			             & \texttt{DE}	&	Extensometry\\
      			             & \texttt{GM}	&	Magnetometry\\
      			             & \texttt{GE}	&	Electric\\
      			             & \texttt{CS}	&	Hot Springs Analysis\\
      			             & \texttt{CG}	&	Gas Analysis\\
      			             & \texttt{CD}	&	DOAS\\
      			             & \texttt{MW}	&	Weather station\\
      			             & \texttt{PJ}	&	Journal Phenomenology\\
      			             & \texttt{PE}	&	Eruption\\
      			             & \texttt{AT}	&	Transmission\\
      			             & \texttt{AB}	&	Buildings\\
\hline
\end{tabular}
\end{center}
\end{table}

% ==================================================================

\begin{table}
\caption{\wokey{\_ARROWSHAPE} key's parameter syntax is a 4-element vector of scalars, coma separated = \wokey{HEADW,HEADL,HEADI,LINEW}. Values are ratios relative to the length of reference scale legend arrow equals 1 (see \wofile{CODE/matlab/arrows.m} for further details).}
\label{arrowshape}
\begin{center}
\begin{tabular}{|c|c|l|}
\hline
\textbf{Parameter} & \textbf{Default value} & \textbf{Comment}\\
\hline
\texttt{HEADW}	& 0.15	& arrow's head width\\
\hline
\texttt{HEADL}	& 0.15	& arrow's head length\\
\hline
\texttt{HEADI}	& 0.12	& arrow's head inside length\\
\hline
\texttt{LINEW}	& 0.03	& arrow's line width\\
\hline
\end{tabular}
\end{center}
\label{nodeidcodes}
\end{table}
