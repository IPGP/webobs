%%%%%%%%%%%%%%%%%%%%%%%%%%%%%%%%%%%%%%%%%%%%%%%%%%%%%%%%%%%%%%%%%%%%
%%%%%%%%%%%%%%%%%%%%%%%%%%%%%%%%%%%%%%%%%%%%%%%%%%%%%%%%%%%%%%%%%%%%

\chapter{Introduction}

\webobs is an integrated web-based system for data monitoring and networks management. Seismological and volcanological observatories have common needs and often common practical problems for multi disciplinary data monitoring applications. In fact, access to integrated data in real-time and estimation of uncertainties are keys for an efficient interpretation, but instruments variety, heterogeneity of data sampling and acquisition systems lead to difficulties that may hinder crisis management. In the Guadeloupe observatory, we have developed in the last 15 years an operational system that attempts to answer the questions in the context of a pluri-instrumental observatory. Based on a single computer server, open source scripts (with few binaries) and a Web interface, the system proposes:

\begin{itemize}
	\item  an extended database for networks management, stations and sensors (maps, station file with log history, technical characteristics, meta-data, photos and associated documents);
	\item  a web-form interfaces for manual data input/editing and export (like geochemical analysis, some of the deformation measurements, …); routine data processing with dedicated automatic scripts for each technique, production of validated data outputs, static graphs on preset moving time intervals, possible e-mail alarms, sensors and station status based on data validity;
	\item  in the special case of seismology, a multichannel continuous stripchart associated with EarthWorm~\footnote{see \url{http://www.isti.com/products/earthworm/}} and SeisComP3~\footnote{see \url{http://www.seiscomp3.org/}} acquisition chain, event classification database, automatic shakemap reports, regional catalog with associated hypocenter maps.
\end{itemize}

\webobs is presently fully functional and used in a dozen observatories, but the documentation is mostly incomplete. We hope to shortly finish the main user’s manual. If you are in a hurry, please contact the project coordinator and we will be happy to help you to install it. \webobs is fully described in the following paper: please cite this one if you publish something using \webobs:

\fbox{\parbox{\textwidth}{\small Beauducel F., D. Lafon, X. Béguin, J.-M. Saurel, A. Bosson, D. Mallarino, P. Boissier, C. Brunet, A. Lemarchand, C. Anténor-Habazac, A. Nercessian, A. A. Fahmi (2020). WebObs: The volcano observatories missing link between research and real-time monitoring, \textit{Frontiers in Earth Sciences}, \textbf{8}(47), \url{https://doi:10.3389/feart.2020.00048}.}}