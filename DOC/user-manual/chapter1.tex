%%%%%%%%%%%%%%%%%%%%%%%%%%%%%%%%%%%%%%%%%%%%%%%%%%%%%%%%%%%%%%%%%%%%
%%%%%%%%%%%%%%%%%%%%%%%%%%%%%%%%%%%%%%%%%%%%%%%%%%%%%%%%%%%%%%%%%%%%
\definecolor{charcoal}{gray}{0.30}
\lstdefinestyle{console}
{language=bash,
basicstyle=\scriptsize\ttfamily\color{white}\bfseries,
backgroundcolor=\color{charcoal},
keywordstyle=\color{white},
alsoletter={:~\$},
}

\chapter{Installation}

% ==================================================================
\section{Overview}

\begin{figure}[!h]
	\centering
	\includegraphics[width=\textwidth]{figures/bigwo.pdf}
	\caption{\webobs big picture}
\end{figure}

% ==================================================================
\section{System requirements}

\subsection{System and package}

\webobs server can run on most \textbf{Linux} systems.

It has been successfully installed/tested on:
\begin{itemize}
\item   Linux 3.13.0 , x86\_64 , Ubuntu 14.04 LTS
\item   Linux 4.13.0 , x86\_64 , Ubuntu 16.04 LTS
\item   Linux 4.15.0 , x86\_64 , Ubuntu 18.04 LTS
\item   Linux 2.6.32, i386 , Debian 2.6.32-48squeeze4
\item   Linux version 3.2.0-4-amd64, Debian 3.2.51-1
\item   Linux version 3.2.0-4-amd64, Debian 4.6.3-14
\item   Linux version 4.9.0-5-amd64, Debian 4.9.65-3
%\item   Mac OS X 10.11.5, Darwin 15.5.0
\end{itemize}

\webobs (browser) clients have been successfully tested with:
\begin{itemize}
\item   FireFox (up to FireFox 93)
\item   Safari (up to Safari 14)
\end{itemize}

\fbox{
	\parbox{\textwidth}{
	Download  the \webobs latest Release and \textbf{MatLab} Runtime Compiler from:\\
	\url{https://ipgp.github.io/webobs/}
	}
}

% -----------------------------------------------------------------
\subsection{Software requirements}

\subsubsection{Required installations}

The following softwares will be tested for existence by the \textbf{setup} installation procedure:

\begin{itemize}
\item   Perl 5.14+ (\textbf{setup} will also check for additional Perl modules)
\item   Apache 2.2+
\item   Sqlite 3.7.9
\item   ImageMagick 6.6.9 (convert+identify)
\item   Mutt 1.5+
\item   MatLab MCR R2011b
\end{itemize}

\subsubsection{Bundled software}

The following lists softwares included in \webobs package:

\begin{itemize}
\item   SeisComP3 (slinktool + arclink\_fetch)
\item   JavaScript extensions:
\begin{itemize}
\item   JQuery
\item   flot
\item   markItUp
\item	MultiMarkDown
\item   overlib
\end{itemize}
\end{itemize}

% ==================================================================
\section{Initial installation}

You must have \textbf{root} privileges to execute \webobs installation. 

\textbf{Installation procedure}:

\begin{itemize}
\item   Choose/create your target \webobs directory, and \wocmd{cd} to it. For
demonstration purposes in this document we will use \wocmd{/opt/webobs/} as the target \webobs directory. 
\item   Download \webobs package and \textbf{MatLab RunTime Compiler}. For
demonstration purposes in this document we will use \wocmd{WebObs-\release.tgz} as the \webobs package. 
\item   Choose/create the system's \webobs user+group (aka \wocmd{WebObs Owner}) if you don't want \wocmd{setup} procedure
to create one itself.

\fbox{
	\parbox{\textwidth}{
	The \webobs \textbf{user}, and its corresponding \textbf{group}, is the required \webobs administration account. 
	It must have a home directory. It will be the owner of the \webobs CONF,LOGS,DATA,WWW,OUTx directories.  
	The Apache http server user will be made a member of \webobs user's group.
	The \webobs user can be used to launch the \webobs Scheduler and Postboard daemons.
	}
}

\item   Untar \webobs package. This will create/populate the \webobs version subdirectory: 
\wocmd{/opt/webobs/WebObs-\release}.
\item   Run the \wocmd{setup} procedure (again, with \wocmd{root} privileges):
\begin{itemize}
\item   must be called as \wocmd{/opt/webobs/WebObs-\release/SETUP/setup} (run \wocmd{setup -h} to see available options)
\item   \wocmd{setup} will gather information/location from your system/environment,
check for some dependencies (see Requirements section above), build the \webobs
structure, optionaly customize Apache's \webobs Virtual Host, set required system's
ownerships and access rights, install systemd services to manage the Postboard
and the Scheduler (on platforms using systemd), and populate your brand new
\webobs with ready-to-use demonstration data and templates.
\item   once completed, \wocmd{setup} will display \webobs configuration (see \wocmd{qsys} command below).
\item   \webobs now ready, you need to activate the \wocmd{scheduler}, \wocmd{postboard} and (re)start Apache http server. 
\end{itemize}
\end{itemize}

\begin{figure}[!h]
	\centering
	\includegraphics[width=\textwidth]{figures/disk_overview.pdf}
	\caption{Overview of \webobs installed disk structure}
\end{figure}

% ==================================================================
\section{Upgrades}

You must have \textbf{root} privileges to execute \webobs upgrades.

The \wocmd{setup} procedure used for Initial Installation, is also used to upgrade \webobs to a new release.
It will automatically detect that an upgrade is intended rather than a first time installation.

\textbf{Upgrade procedure:}

\begin{itemize}
\item   Download \webobs package corresponding to the version you want to upgrade to.
\item   Untar \webobs package.
\item   Run the \wocmd{setup} procedure by running the script \wocmd{/opt/webobs/WebObs-\release/SETUP/setup} (use the \wocmd{-h} option to see the help screen).
\item   \wocmd{setup} reports Version changes and important information in the
\wocmd{SETUP.CONF.README} file. Please read it carefully.
\item   if you choose to do so, \wocmd{setup} will also append to the
	configuration files of your installation (WebObs, Scheduler, and Grids
	configuration files, etc.) the configuration variables introduced by the
	current release and let you edit all the updated files to review or adapt
	the changes.
\end{itemize}

% ==================================================================
\section{qsys - query configuration}

\wocmd{qsys} script, also automatically executed when \wocmd{setup} ends, will
display your base \webobs configuration.

\begin{lstlisting}[style=console,title=\wofile{qsys} example]
                                     __ __ __
             ...oooooWoooooo..     W \ V  V /          Id: ipgp
          .oo.. ............o..o.. W  \_/\_/      Version: WebObs-2.0.0
       .oo.     ..............  ..oW   ___ 
     .oo.      ...............     W  / -_)         Owner: webobs[webobs]
    .W..     .  .............o.... W  \___|          Root: /opt/webobs
   .W.............o.........oWWWoo.W   _           Config: /opt/webobs/CONF/WEBOBS.rc
  .W    .   .  . ............o.....W  | |__          Logs: /opt/webobs/LOGS
  W.        ..   .WWWWo....  .     W  | '_ \
 .W        .Wo. .Wxxxxx....        W  |_.__/          MCR: /usr/local/MATLAB/MATLAB_Compiler_Runtime/v7...
 .W        .WWW.WxxxxxxW...   ...  W   ___        Rawdata: /opt/rawdata
 .W       .WWWWWxxxxxxxxW..........W  / _ \        Sefran: /opt/sefran
  Wo..ooWxxxxxxWxxxxxxxxxxo........W  \___/
  .xxxxxxxxxxxxxxWxxxxxxxxxW.......W   _             HTTP: http://webobs.ipgp
   .xxxxxxxxxxxxxWWWWWxxxxxxW.....oW  | |__                /etc/apache2/sites-available/webobs
    .WxxxxxxxxxWxxWWWWWWWxxxxxWWo..W  | '_ \               running on Apache/2.2.22 (Ubuntu)
     .oxxxxxxxxxxxxxWWWWWWWxxxxxxxWW  |_.__/
       .oWxxxxxxxxxxxxxWWWWWWWWxxxWW   ___      Scheduler: not running
          .oWxxxxxxxxxxxxWWWWWxW.. W  (_-<      PostBoard: not running
             ...oWWWxxxWWWWo..     W  /__/
                                                                qsys run 2014-08-08 12:05:06

\end{lstlisting}

% ==================================================================
\section{Configuration files syntax}

\webobs configuration files (*.rc, *.conf, or *.cnf) used to customize your installation
and described in this document, typically define one 
functionnal parameter (\wocmd{key}) per line, made up of one or more associated values
(\wocmd{fields}). They can share the same set set of syntactic rules for  
parsing/interpretation:

1) In order to be parsed/interpreted according to the following rules, the files must
contain a so-called 'definition line' (identified with \wocmd{=} in column 1) as the 
first interpreted line. This definition line is also used to further define parsing, 
that comes in two (2) flavors:

\begin{tabular}{ll}
\wocmd{=key\textbar value}  &  one value per key (Perl's equiv. \$X\{key\} =\textgreater value)\\
\wocmd{=key\textbar name1\textbar ...\textbar nameN}  & multiple named values per key (Perl's equiv. \$X\{key\}\{name1\} =\textgreater value)\\
\end{tabular}

2) Any text following a \wocmd{\#} is considered a comment and discarded.

3) Blank lines are discarded, leading and trailing blanks too.

4) Fields separator character, within interpreted lines, is \textbar (pipe).

5) \wocmd{\textbar} and \wocmd{\#} characters that must belong to a field value may be 
'escaped' (ie. not interpreted as separator or comment respectively) by prefixing them 
with a \textbackslash .

6) Field value substitution (interpolation) is allowed in \wocmd{=key\textbar value} format:

\begin{tabular}{ll}
\wocmd{\$\{key\} in value}       & will be replaced with the value of the key \textbar value pair of the current file.\\
\wocmd{\$WEBOBS\{key\} in value} & will be replaced with the value of the \webobs main configuration \wocmd{key} value. \\
\end{tabular}


% ==================================================================
\section{\webobs tree}
%\begin{table}[p]
\begin{center}
    \begin{longtable}{ll}
		\fcolorbox[gray]{0.1}{0.9}{\wocmd{CONF/}} & configurations  \\
	    \hspace{0.4cm} \wocmd{WEBOBS.rc}             & main \webobs configuration file  \\
	    \hspace{0.4cm} \wocmd{*.\{rc,conf,cnf\}}     & other configuration files \\
	    \hspace{0.4cm} \wocmd{FORMS/}                & FORMS definitions \\
	    \hspace{0.8cm} \wocmd{formname/}             & definitions for FORM formname    \\
	    \hspace{0.4cm} \wocmd{PROCS/}                & PROCS definitions \\
	    \hspace{0.8cm} \wocmd{procname/}             & definitions for PROC procname    \\
	    \hspace{0.4cm} \wocmd{SEFRANS/}              & SEFRANS definitions \\
	    \hspace{0.8cm} \wocmd{sefranname/}           & definitions for SEFRAN sefranname    \\
	    \hspace{0.4cm} \wocmd{VIEWS/}                & VIEWS definitions \\
	    \hspace{0.8cm} \wocmd{viewname/}             & definitions for VIEW viewname    \\
	    \hspace{0.4cm} \wocmd{GRIDS2FORMS/}          & links from PROCs to FORMS  \\
	    \hspace{0.8cm} \wocmd{PROC.pname.formname}   & -\textgreater ../formname  \\
	    \hspace{0.4cm} \wocmd{GRIDS2NODES/}          & links from GRIDs to NODES  \\
	    \hspace{0.8cm} \wocmd{PROC.pname.nodename}   & -\textgreater ../../DATA/NODES/nodename  \\
	    \hspace{0.8cm} \wocmd{VIEW.vname.nodename}   & -\textgreater ../../DATA/NODES/nodename  \\
	    \hspace{0.4cm} \wocmd{MENUS/}                & group's menus \\
	    \hspace{0.8cm} \wocmd{+GID}                  & menu content for group +GID \\
	    \\
		\fcolorbox[gray]{0.1}{0.9}{\wocmd{CODE/}} & \webobs code  \\
	    \hspace{0.4cm} \wocmd{bin/}                  & executables                \\
	    \hspace{0.4cm} \wocmd{cgi-bin/}              & Perl CGIs                  \\
	    \hspace{0.4cm} \wocmd{css/}                  & HTML Style Sheets          \\
	    \hspace{0.4cm} \wocmd{html/}                 & static HTML pages          \\
	    \hspace{0.4cm} \wocmd{icons/}                & HTML icons, static images  \\
	    \hspace{0.4cm} \wocmd{js/}                   & javascript                 \\
	    \hspace{0.4cm} \wocmd{matlab/}               & Matlab codes and library   \\
	    \hspace{0.4cm} \wocmd{python/}               & Python codes and library   \\
	    \hspace{0.4cm} \wocmd{shells/}               & bash commands              \\
	    \hspace{0.4cm} \wocmd{tplates/}              & configurations templates   \\
	    \\
		\fcolorbox[gray]{0.1}{0.9}{\wocmd{DATA/}} & data \\
	    \hspace{0.4cm} \wocmd{DB/}                  & built-in tools data         \\
	    \hspace{0.8cm} \wocmd{*.DAT}                & \\
	    \hspace{0.4cm} \wocmd{DEM/}                 & Digital Elevation Model files \\
	    \hspace{0.4cm} \wocmd{NODES/}               & NODES configurations and data \\
		\hspace{0.8cm} \wocmd{nodename/}            & nodename \\
		\hspace{1.2cm} \wocmd{*.txt}                & system features descriptions \\
		\hspace{1.2cm} \wocmd{*.cnf}                & configuration                \\
		\hspace{1.2cm} \wocmd{*.clb}                & calibration file             \\
		\hspace{1.2cm} \wocmd{DOCUMENTS/}           & documents                    \\
		\hspace{1.2cm} \wocmd{FEATURES/}            & features                     \\
		\hspace{1.2cm} \wocmd{EVENTS/}              & events log                   \\
		\hspace{1.2cm} \wocmd{PHOTOS/}              & pictures                     \\
	    \\
		\fcolorbox[gray]{0.1}{0.9}{\wocmd{DOC/}} & documentations  \\
	    \\
		\fcolorbox[gray]{0.1}{0.9}{\wocmd{LOGS/}} & tools, appl, daemons logs  \\
	    \\
		\fcolorbox[gray]{0.1}{0.9}{\wocmd{OUTG/}} & batch Procs Outputs  \\
		\hspace{0.4cm} \wocmd{PROC.procname/}       & PROC procname outputs \\
		\hspace{0.8cm} \wocmd{exports/}             &  \\
		\hspace{0.8cm} \wocmd{graphs/}              &  \\
		\hspace{0.8cm} \wocmd{maps/}                &  \\
		\hspace{0.8cm} \wocmd{events/}              &  \\
	    \\
		\fcolorbox[gray]{0.1}{0.9}{\wocmd{OUTR/}} & Procs Requests Outputs  \\
		\hspace{0.4cm} \wocmd{20140720\_084340\_85.199.99.199\_userX/}    & Dated Request  \\
		\hspace{0.8cm} \wocmd{REQUEST.rc}          &  Request parameters \\
		\hspace{0.8cm} \wocmd{PROC.procname/}      &  Requested Proc outputs \\
		\hspace{1.2cm} \wocmd{graphs/}             &   \\
		\hline
    \end{longtable}
\end{center}
%\end{table}


% ==================================================================
\section{License}

